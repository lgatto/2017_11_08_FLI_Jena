\section{Spatial proteomics}

\subsection{The LOPIT pipeline}

\subsection*{Introduction}
\label{sec:spintro}

\begin{frame}{Regulations}
  \begin{figure}[h]
    \centering
    \includegraphics[width=1\linewidth]{Figures2/regulation.jpg}
  \end{figure}
\end{frame}

\label{sec:spspatprot}

\begin{frame}{Cell organisation}
  \begin{center}
    \includegraphics[width=1\linewidth]{Figures/Animal_cell_structure.png} \\
    \textbf{\textcolor{Blue}{Spatial proteomics}} is the systematic
    study of protein localisations.
  \end{center}

  \tiny Image from Wikipedia
  \url{http://en.wikipedia.org/wiki/Cell_(biology)}.
\end{frame}

% \begin{frame}{Spatial proteomics - Why?}
%   \begin{itemize}
%   \item Localisation is function
%   \item Mis-localisation \citep{Kau2004,Laurila2009}
%   \item Re-localisation
%   \end{itemize}
% \end{frame}
 
\begin{frame}{Spatial proteomics - Why?}
  \begin{block}{Localisation is function} 
    \begin{itemize}
    \item The cellular sub-division allows cells to establish a range
      of distinct micro-environments, each favouring different
      biochemical reactions and interactions and, therefore, allowing
      each compartment to fulfil a particular functional role.          
    \item Localisation and sequestration of proteins within
      sub-cellular niches is a fundamental mechanism for the
      post-translational regulation of protein function.         
    \end{itemize}
  \end{block}    
\end{frame}

\begin{frame}{Spatial proteomics - Why?}
  \begin{block}{Mis-localisation}   
    Disruption of the targeting/trafficking process alters proper
    sub-cellular localisation, which in turn perturb the cellular
    functions of the proteins.  
    \begin{itemize}
    \item Abnormal protein localisation leading to the loss of functional
      effects in diseases \citep{Laurila2009}.    
    \item Disruption of the nuclear/cytoplasmic transport (nuclear
      pores) have been detected in many types of carcinoma cells
      \citep{Kau2004}.
    \end{itemize}      
  \end{block}

  \begin{block}{Re-localisation in} 
    \begin{itemize}
    \item \textcolor{Blue}{Differentiation}: Tfe3 in mouse ESC
      \citep{Betschinger:2013}.
    \item \textcolor{Blue}{Metabolism}: changes in carbon sources, elemental
      limitations.
    \end{itemize}
  \end{block}
\end{frame}


\subsubsection*{Experimental designs}
\label{sec:expdesign}

\begin{frame}{Spatial proteomics - How, experimentally}
  \begin{figure}
    \includegraphics[width=.8\linewidth]{Figures/F02-expdesigns.pdf}
    \caption{Organelle proteomics approaches \citep{Gatto:2010}}
  \end{figure}
\end{frame}


% \begin{frame}{Microscopy}
%   \begin{columns}[t]
%     \begin{column}[T]{0.5\textwidth}      
%       \begin{figure}
%         \includegraphics[width=.7\linewidth]{Figures2/Localisations02eng.jpg}
%         \caption{\textbf{Fluorescent protein fusion} localisation of
%         proteins using GFP tagging. (from Wikipedia)}
%       \end{figure}
%     \end{column}
%     \begin{column}[T]{0.49\textwidth}
%       \begin{centering}
%         \begin{figure}
%           \includegraphics[width=.49\linewidth]{Figures2/172H11blueredgreen.jpg} 
%           \includegraphics[width=.49\linewidth]{Figures2/34C71blueredgreen.jpg}
%           \caption{\textbf{Immunofluorescence}: ZFPL1, GO (left) and FHL2, mainly localized to
%           actin filaments and focal adhesion sites. Also detected in
%           the nucleus (right). (from the Human Protein Atlas)}
%         \end{figure}
%       \end{centering}      
%     \end{column}
%   \end{columns}        
% \end{frame}

\begin{frame}{Fusion proteins and immunofluorescence}

  \begin{figure}[h]
    \centering
    \includegraphics[width=.35\linewidth]{Figures2/Localisations02eng.jpg}        
    \includegraphics[width=.45\linewidth]{figures/if_selected.jpg}    
    \caption{Targeted protein localisation.}
  \end{figure}
\end{frame}

\begin{frame}{Fusion proteins and immunofluorescence}
  \begin{figure}
    \centering
    \includegraphics[angle=-90, width=.8\linewidth]{Figures2/Disc-IF-IP.png} \\
    \includegraphics[angle=-90, width=.8\linewidth]{Figures2/Disc-N-C.png}
    \caption{Example of discrepancies between IF and FPs as well as
      between FP tagging at the N and C termini (Stadler et al., 2013).}
  \end{figure}
\end{frame}

\subsubsection*{Gradient approaches}
\label{sec:grad}

\begin{frame}{Spatial proteomics - How, experimentally}
  \begin{figure}
    \includegraphics[width=.8\linewidth]{Figures/F02-expdesigns.pdf}
    \caption{Organelle proteomics approaches
      \citep{Gatto:2010}. Gradient approaches: \cite{Dunkley:2006},
      \cite{Foster2006}.}
  \end{figure}
  $\Rightarrow$ \textbf{Explorative/discovery approches},
  \textcolor{Blue}{\textbf{global localisation maps}}.
\end{frame}


\begin{frame}{}
  \begin{figure}
    % \includegraphics[width=.8\linewidth]{Figures/F03-protocols-8plex.pdf}
    % \includegraphics[width=.5\linewidth]{figures/expdesign.pdf}
    \includegraphics[width=.39\linewidth]{figures/workflow_primary.pdf}
  \end{figure} 
\end{frame}

\subsection*{Data analysis}
\label{sec:comp}

\subsubsection*{The data}
\label{sec:data}
 
\begin{frame}{Quantitation data and organelle markers}
  \begin{center}
    \begin{tabular}{|l|llll||l|}
      \hline
      & Fraction$_{\text{1}}$ & Fraction$_{\text{2}}$ & \ldots{} & Fraction$_{\text{m}}$ & markers\\
      \hline
      p$_{\text{1}}$ & q$_{\text{1,1}}$ & q$_{\text{1,2}}$ & \ldots{} & q$_{\text{1, m}}$ & unknown \\
      p$_{\text{2}}$ & q$_{\text{2,1}}$ & q$_{\text{2,2}}$ & \ldots{} & q$_{\text{2, m}}$ & \textcolor{Red}{$loc_{1}$}\\
      p$_{\text{3}}$ & q$_{\text{3,1}}$ & q$_{\text{3,2}}$ & \ldots{} & q$_{\text{3, m}}$ & unknown \\
      p$_{\text{4}}$ & q$_{\text{4,1}}$ & q$_{\text{4,2}}$ & \ldots{} & q$_{\text{4, m}}$ & \textcolor{Blue}{$loc_{i}$}\\
      \vdots & \vdots & \vdots & \vdots & \vdots & \vdots\\
      p$_{\text{j}}$ & q$_{\text{j,1}}$ & q$_{\text{j,2}}$ & \ldots{} & q$_{\text{j, m}}$ & unknown \\
      \hline
    \end{tabular}
  \end{center}
\end{frame}

\subsubsection*{Visualisation}
\label{sec:viz}

\begin{frame}{Visualisation and classification}
  \begin{figure}
    \centering
    \includegraphics[width=.6\linewidth]{Figures/F04-analyses.pdf}
    \caption{From \cite{Gatto:2010}, \textit{Arabidopsis thaliana} data
      from \cite{Dunkley:2006}}
  \end{figure}   
\end{frame}

\subsubsection*{Machine learning}
\label{sec:ml}

\begin{frame}{Data analysis}
  \begin{centering}
    \raisebox{-0.5\height}{\includegraphics[width=.57\linewidth]{./Figures2/Fig1-data-a.pdf}}
    \raisebox{-0.5\height}{\includegraphics[width=.41\linewidth]{./Figures2/Fig1-data-b.pdf}}

    \begin{block}{Supervised machine learning}    
      Using labelled marker proteins to match unlabelled proteins (of
      unknown localisation) with similar profiles and classify them as
      residents to the markers organelle class.
    \end{block}
  \end{centering}
\end{frame}


\begin{frame}{Supervised ML}
  \begin{figure}[h]
    \centering 
    \includegraphics[width=\linewidth]{figs_local/hyperlopit-class.pdf}
    \caption{Support vector machines classifier on the embryonic stem
      cell data from \cite{Christoforou:2016}.}
  \end{figure}
\end{frame}


\begin{frame}{Limitations}
  \begin{columns}[t]
    \begin{column}[T]{0.43\textwidth}
      \begin{centering}
        \includegraphics[width=1\linewidth]{Figures/tan2009r1org.pdf}
      \end{centering} 
    \end{column}
    \begin{column}[T]{0.56\textwidth}      
      \includegraphics[width=1\linewidth]{Figures/Animal_cell_structure.png}
    \end{column}
  \end{columns}          
  Incomplete annotation, and therefore lack of training data, for
  many/most organelles. \textit{Drosophila} data from \cite{Tan2009}.
\end{frame}

\begin{frame}{Novelty detection}
  \begin{figure}
    \includegraphics[width=.48\linewidth]{Figures/tan2009r1org.pdf}
    \includegraphics[width=.5\linewidth]{Figures/pdres2fig.pdf}
    \caption{Left: \textit{Drosophila} data from
      \cite{Tan2009}. Right: Semi-supervised learning,
      \cite{Breckels:2013}.}
  \end{figure} 
\end{frame}

% \begin{frame}
%   \centering    
%   \begin{columns}[t]
%     \begin{column}[T]{0.4\textwidth}
%       \begin{figure}[h]    
%         \includegraphics[width=.9\linewidth]{Figures/tan2009r1org.pdf} \\
%         \includegraphics[width=.9\linewidth]{Figures/pdres2fig.pdf}
%       \end{figure}
%     \end{column}
%     \begin{column}[T]{0.59\textwidth}         
%       \begin{figure}[h]    
%         \includegraphics[width=.9\linewidth]{Figures/phenodisco.pdf} 
%       \end{figure}
%     \end{column}
%   \end{columns}
%   % \caption{The \texttt{phenoDisco} algorithm \citep{Breckels:2013}.}
% \end{frame}

\subsection{Improving on LOPIT}

\begin{frame}{Improving on LOPIT}

  Improving is obtaining better sub-cellular resolution to increase
  the number of protein that can be \textbf{confidently} assigned to a
  sub-cellular niche.

  \begin{figure}[h]
    \centering
    \includegraphics[width=1\linewidth]{/home/lg390/Pictures/Figures/E14-lopit-hyperlopit.pdf}    
    \caption{E14TG2a embryonic stem cells: old (left) \textit{vs.} new, better
      resolved (right) experiments (\cite{Christoforou:2016}).}
  \end{figure}
  
\end{frame}  

\begin{frame}{Improving on LOPIT}
  \centering
  \begin{tabular}{| p{5cm} | p{5cm} |}
    \hline
    \makecell{LOPIT\\ \cite{Dunkley:2006}}    & \makecell{\textbf{Computational}:\\ \textit{transfer learning}\\ \cite{Breckels:2016}} \\
    \hline
    \makecell{\textbf{Experimental}:\\ \textit{hyperLOPIT}\\ \cite{Christoforou:2016} \\ \cite{Mulvey:2017}} & \makecell{Biological\\discoveries}  \\
    \hline    
  \end{tabular}
\end{frame}

\subsubsection{Experimental advances: hyperLOPIT}

\begin{frame}{hyperLOPIT}
  \begin{figure}[h]
    \centering
    \includegraphics[width=.8\linewidth]{/home/lg390/Pictures/Figures/nprot-hyperlopit-2017-026-F1.jpg}
    \caption{From \cite{Mulvey:2017} \textit{Using hyperLOPIT to
        perform high-resolution mapping of the spatial proteome}.}
    \label{fig:hyperlopit}
  \end{figure}
\end{frame}

\begin{frame}
  \begin{figure}[h]
    \centering
    \includegraphics[width=1\linewidth]{/home/lg390/Pictures/Figures/E14-lopit-hyperlopit-rep1.pdf}
    \caption{E14TG2a LOPIT on 8 fractions (using iTRAQ 8-plex) and
      1109 proteins \textit{vs.}  hyperLOPIT on 10 fractions (using
      TMT 10-plex) and SPS-MS$^3$ for 5032 proteins.}
  \end{figure}

\end{frame}

\subsubsection{Computational advances: Transfer learning}

\begin{frame}{Transfer learning}
  What about annotation data from repositories such as the Gene Ontogy
  (GO), sequence features, signal peptide, transmembrane domains,
  images, prediction software, \ldots

  \begin{block}{}
    \begin{itemize}
    \item From a \underline{user perspective}: \textbf{"free/cheap"}
      vs. expensive
    \item Abundant (all proteins, 100s of features) vs. (experimentally)
      limited/\textbf{targeted} (1000s of proteins, 6 -- 20 of features)
    \item For localisation in \underline{system at hand}: \textit{low}
      vs. high \textbf{quality}
    \item \textbf{Static} vs. \textbf{dynamic}
    \end{itemize}
  \end{block}

\end{frame}

\begin{frame}{Transfer learning}
  What about annotation data from repositories such as the
  \underline{Gene Ontology (GO)}, sequence features, signal peptide,
  transmembrane domains, images, prediction software, \ldots

  \begin{block}{Transfer learning}
    Support/complement the \textbf{primary} target domain
    (experimental data) with \textbf{auxiliary} data (annotation,
    imaging, PPI, ...)  features without compromising the integrity of
    our primary data \citep{Breckels:2016}.
  \end{block}

\end{frame}


\begin{frame}
  \begin{center}
    \includegraphics[width=.7\linewidth]{figures/workflow.pdf}      
  \end{center}
\end{frame}

\begin{frame}
  \textbf{Transfer learnig}, based on \cite{Wu:2004}:
  \begin{columns}[t]
    \begin{column}[T]{0.5\textwidth}            
      \begin{block}{Class-weighted kNN}      
        \begin{equation*}
          \label{eq:translearn}
          V(c_i)_j = \theta^* n_{ij}^{P} + (1 - \theta^*) n_{ij}^{A}
        \end{equation*}
      \end{block}
    \end{column}
    \begin{column}[T]{0.4\textwidth} 
      \includegraphics[width=1\linewidth]{figures/pca-lopit-andy-e14.pdf} \\
    \end{column}    
  \end{columns}
  \begin{block}{Linear programming SVM}
    \begin{equation*}\label{eq:translearn2}
      f(\bx, \bv; \bal_P, \bal_A, b) = \sum_{l=1}^m y_l \left[ \alpha_l^P
        K^P(\bx_l, \bx) + \alpha_l^A K^A(\bv_l, \bv) \right] + b
    \end{equation*}
  \end{block}
\end{frame}


% \include{indtransalgo}

% \begin{frame}{Internally (searching for best $\theta$)}
%   \begin{figure}
%     \centering
%     \includegraphics[width=\linewidth]{figures/smooth-f1-weights.pdf}    
%     \caption{Left: Mouse stem cells (e14tg2a). Right: Human Embryonic
%     Kidney 293 (HEK293)}
%   \end{figure}
% \end{frame}


\begin{frame}
  % \includegraphics[width=.6\linewidth]{figures/thetaByClass.pdf}
  % \includegraphics[width=.39\linewidth]{figures/bubbleWeights.pdf}
  \includegraphics[width=1\linewidth]{figures/e14.pdf}
  \\
  \scriptsize
  Data from mouse stem cells (E14TG2a).
\end{frame}


\begin{frame}{}

  \begin{figure}[h]
    \centering
    \includegraphics[width=.8\linewidth]{/home/lg390/Pictures/Figures/E14-lopit-hyperlopit.pdf} \\
    \includegraphics[width=.4\linewidth]{/home/lg390/Pictures/Figures/2016-PLoSCB-TL-classifierDiscriminationPowerk5.pdf}
    \includegraphics[width=.4\linewidth]{/home/lg390/Pictures/Figures/2016-PLoSCB-TL-roc-new.pdf}    
    \caption{{\footnotesize From \cite{Breckels:2016} \textit{Learning from
          heterogeneous data sources: an application in spatial
          proteomics}.}
    }
\label{fig:tlres}
  \end{figure}
  
\end{frame}

\subsection{Biological applications}

\begin{frame}{Biological applications}
  \begin{itemize}
  \item Multi-localisation
  \item Trans-localisation
  \end{itemize}
  Dependent on good sub-cellular resolution.
\end{frame}

\subsubsection{Dual-localisation}

\begin{frame}
  \textcolor{Blue}{\textbf{Dual-localisation}} Proteins may be present
  simultaneously in several organelles (e.g. trafficking). Example
  from embryonic stem cells \citep{Christoforou:2016}.
  \begin{columns}
    \begin{column}{.5\textwidth}
      \includegraphics[width=1\linewidth]{figures/dual-loc.pdf}<+->
    \end{column}
    \begin{column}{.5\textwidth}<+->
      \centering
      \includegraphics[width=.8\linewidth]{figures/Tfe3.png}\\
      \tiny From \cite{Betschinger:2013} \\
      \includegraphics[width=1\linewidth]{figures/Tfe3.pdf}
      Example from 
    \end{column}  
  \end{columns}
\end{frame}

\subsubsection{Trans-localisation}

\begin{frame}{Spatial dynamics}
  \begin{block}{Trans-localisation monocyte to macrophage
      differenciation}
    Goal: to investigate the effect of LPS-mediated inflammatory
    response in human monocytic cells (THP-1)    
  \end{block}

  \begin{block}{Data}
    \begin{itemize}
    \item Triplicate \textbf{temporal} profiling (0, 2, 4, 6, 12, 24
      hours).
    \item Triplicate \textbf{spatial} profiling (0 vs 12 hours) -
      early trafficking components, before actual morphological
      differentiation at 24h.
    \end{itemize}
  \end{block}

  Work lead by \textbf{Dr Claire Mulvey}, Cambridge Centre for
  Proteomics.
  
\end{frame}



\begin{frame}
  \begin{figure}[h]
    \centering
    \includegraphics[width=\linewidth]{./figs_local/lps.pdf}
    \caption{Spatial maps: unstimulated and LPS-treated.}
  \end{figure}
\end{frame}

\begin{frame}
  \begin{figure}[h]
    \centering
    \includegraphics[width=\linewidth]{./figs_local/lps-pkc.pdf}
    \caption{Relocation of Protein Kinase C alpha and beta from the
      cytosol to the plasma membrane, driving maturation into a
      differentiated macrophage phenotype.}
  \end{figure}
\end{frame}

\begin{frame}
  \begin{figure}[h]
    \centering
    \includegraphics[width=\linewidth]{./figs_local/lps-stat.pdf}
    \caption{Relocation of STAT6 from the cytosol to the Nucleus,
      activating anti-bacterial and anti-viral-like
      response. Validated by microscopy and see also
      \cite{Chen:2011}.}
  \end{figure}
\end{frame}



\begin{frame}{Beyond organelles: application to PPI/Protein complexes}

  \begin{figure}[t]
    \includegraphics[width=.6\linewidth]{figures/knn-x1f-1.pdf}
    \caption{Data on proteasome complexes from Fabre \textit{et
        al.}  Mol Syst Biol (2015), DOI:
      \url{10.15252/msb.20145497}}
  \end{figure}

\end{frame}


